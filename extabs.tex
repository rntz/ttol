\documentclass[11pt]{article}

\usepackage[letterpaper,top=1in,bottom=1in,left=1in,right=1in]{geometry}

\usepackage{amsmath,amssymb}
\usepackage{array}
\usepackage{color}
\usepackage{enumerate}
\usepackage{latexsym}           %for \leadsto
\usepackage{url}

\newcommand{\bscolor}{red}
\newcommand{\bs}[1]{\textcolor{\bscolor}{#1}}

\title{A Type Theory for Linking\\{\large (extended abstract)}}
\author{Michael Arntzenius}

\begin{document}

\maketitle

{\small NB. Parts of this paper which are as yet unproven and/or unimplemented
  are in \bs{\bscolor}.}

% TODO: abstract goes here.

\section{Audience}

This paper is written assuming a working knowledge of the basics of programming
language syntax and semantics. Backus-Naur Form (BNF) is used to define grammars
and natural-deduction style inference rules are used to define logical
judgments. Familiarity with typed $\lambda$-calculi, in particular features such
as product, universal, and recursive types, is assumed. Other concepts, such as
modal logic, adjoint logic, and hereditary substitution, are explained briefly
before their use. Prior knowledge of these is not necessary.

% FIXME: Note re understanding linking & loading.

% Detailed knowledge of the usual implementations of the processes of linking and
% loading programs is not required; the purpose of these processes, however,
% informs the motivation of this paper.

% % FIXME: phrasing
% Knowledge of the purpose (though not the implementation) of the processes of
% linking and loading programs is required to understand the motivation of this
% paper.


\section{Motivation}

\section{Background}

\section{$\lambda$-calculi}

\section{Extended CAM}

\section{\bs{Implementation}}

\bs{This entire section is speculative.}

\end{document}
